% ---
% Epígrafe
% Texto vindo do modelo canônico ABNTeX 2
% Altere-o livremente.
% ---
\begin{epigrafe}
    \vspace*{\fill}
  %EPÍGRAFE, CONFORME A NORMA NBR14724/2011 NÂO POSSUI TÍTULO
  
    %APAGAR ESTA PARTE NO MODELO FINAL:
    %%%%%%%%%%%%%%%%%%%%%%%%%%%%%%%%%%%%%%%%%%%%
Epígrafe é opcional. Caso seja inserida no texto, ela deve ser elaborada conforme a ABNT NBR 10520. Deve ser inserida após os agradecimentos. Podem também constar epígrafes nas folhas ou páginas de abertura das seções primárias, ou seja, ao início de cada capítulo da monografia. O tipo de letra da epígrafe pode ter algum tipo de efeito “artístico”.

Caso não seja feita epígrafe, esta página deverá ser excluída. Normalmente a epígrafe aparecerá apenas na versão final do texto enviado para a banca examinadora, sendo desnecessária em etapas intermediárias, como por exemplo os Seminários de Acompanhamento de Projetos (SAP) ou Qualificação para o Doutorado.

A definição da norma ABNT NBR 14724:2011 é “texto em que o autor apresenta uma citação, seguida de indicação de autoria, relacionada com a matéria tratada no corpo do trabalho” (item 3.14).

Veja um exemplo a seguir:
%%%%%%%%%%%%%%%%%%%%%%%%%%%%%%%%%%%%%%%%%%%%%%%%
	\begin{flushright}
		\textit{`A linguagem, tanto falada quanto escrita, é viva, dinâmica e sofre adaptações ou aperfeiçoamentos ao longo do seu uso. A linguagem técnica, entretanto, é tida como perene, sendo, em muitos casos, claramente rechaçadas tentativas de inserção de novos termos em algumas áreas clássicas. Não se pode afirmar de maneira inconteste que esse conceito seja irrestrito, posto que muitas áreas da ciência dependem e evoluem com as novidades ou inovações tecnológicas. A metrologia, nesse contexto, embora seja uma ciência clássica, vem se desenvolvendo calcando-se em aprimoramentos tecnológicos, o que lhe transfere um dinamismo natural dos termos e conceitos empregados.
(OLIVEIRA; COSTA-FELIX, 2017, p. 45).
}
	\end{flushright}
\end{epigrafe}
% ---